\documentclass[9pt]{beamer}
\usetheme{Frankfurt}
\usepackage{multicol}
\usepackage{graphicx}
\linespread{1.35}
\usepackage{amsmath}
\usepackage{color}
\usepackage{xcolor}
\usepackage{tikz}
\usetikzlibrary{arrows,automata}


\begin{document}
\begin{frame}

\section*{Ambiguity in Context-free Grammar }
\begin{flushright}
 \texttt{Context-free Grammar} \hspace*{0.10cm}\textbf{$|$} \textbf{305}\hspace*{0.5cm}
\end{flushright}

\vspace*{1cm}
\fcolorbox{white}{black}{\textbf{\textcolor[rgb]{1.00,1.00,1.00}{Example 6.13}}}\hspace*{0.1cm} \texttt{Check whether the grammar is ambiguous or not.}\\

\vspace*{0.3cm}
\hspace*{4cm} $S \rightarrow aS/AS/A$ \\
\hspace*{4cm} $A \rightarrow AS/a$ \\

\vspace*{0.5cm}
\textbf{Solution:} Consider the string 'aaa'. The string can be generated in many ways. Here, we are giving two
ways.\\
\vspace*{0.2cm}
\hspace*{0.5cm} $i) S \rightarrow aS \rightarrow aAS \rightarrow aaS \rightarrow aaA \rightarrow aaa$ \\
\hspace*{0.5cm} $ii) S \rightarrow A \rightarrow AS \rightarrow ASS \rightarrow aAS \rightarrow aaS \rightarrow aaA \rightarrow aaa$ \\
\end{frame}

\begin{frame}
The parse trees for derivation (i) and (ii) are shown in Figs. 6.6 (i) and (ii).\\

\begin{center}
\section{picture}
\includegraphics[width=8cm,height=5cm]{305.png}
\end{center}

\hspace*{0.5cm} Here, for the same string derived from the same grammar we are getting more than one parse tree.
So, according to the definition, the grammar is an ambiguous grammar.\\

\vspace*{0.5cm}
\end{frame}

\begin{frame}
\fcolorbox{white}{black}{\textbf{\textcolor[rgb]{1.00,1.00,1.00}{Example 6.14}}}\hspace*{0.1cm} \texttt{Prove that the following grammar is ambiguous.}

\hspace*{4cm} $V_N: \{S\}$ \\
\hspace*{4cm} $\Sigma: \{id, +, *\}$ \\
\hspace*{4cm} $P: S \rightarrow S + S/S * S/id$ \\
\hspace*{4cm} $S: \{S\}$ \\

\textbf{Solution:} Let us take a string $id + id*id$.\\
The string can be generated in the following ways.\\

\vspace*{0.2cm}
\begin{flushleft}
\section{picture}
\includegraphics[width=8cm,height=0.8cm]{305-2.png}
\end{flushleft}

\vspace*{0.2cm}
\hspace*{0.5cm} The parse trees for derivation (i) and (ii) are shown in Figs. 6.7 and 6.8.\\
\end{frame}


\begin{frame}
\section*{Transitional Format}
\begin{flushleft}
    \textbf{306}\hspace*{0.1cm} \textbf{$|$} \hspace*{0.1cm} \texttt{Introduction to Automata Theory, Formal Languages and Computation}
  \end{flushleft}

\vspace*{0.5cm}
\begin{center}
\section{picture}
\includegraphics[width=11cm,height=5cm]{306.png}
\end{center}

\vspace*{0.3cm}
\end{frame}

\begin{frame}
\hspace*{0.5cm} As we are getting two parse trees for generating a string from the given grammar, the grammar is
ambiguous.\\

\fcolorbox{white}{black}{\textbf{\textcolor[rgb]{1.00,1.00,1.00}{Example 6.15}}}\hspace*{0.1cm} \texttt{Prove that the following grammar is ambiguous.}\\

\vspace*{0.1cm}
\hspace*{4cm} $S \rightarrow a/abSb/aAb$ \\
\hspace*{4cm} $A \rightarrow bS/aAAb$ \\

\vspace*{0.2cm}
\textbf{Solution:} Take a string abababb. The string can be generated in the following ways.\\

\begin{flushleft}
  \section{picture}
\includegraphics[width=8cm,height=0.8cm]{306-2.png}
\end{flushleft}

\end{frame}

\begin{frame}
\hspace*{0.5cm} The parse trees for (i) and (ii) are given in Fig. 6.9.\\

\begin{center}
\section{picture}
\includegraphics[width=8cm,height=5cm]{306-3.png}
\end{center}

\hspace*{0.5cm} As we are getting two parse trees for generating a string from the given grammar, the grammar is
ambiguous.\\
\end{frame}

\begin{frame}
\section*{Ambiguity in Context-free Grammar }
\begin{flushright}
 \texttt{Context-free Grammar} \hspace*{0.10cm}\textbf{$|$} \textbf{307}\hspace*{0.5cm}
\end{flushright}

\vspace*{0.5cm}
\fcolorbox{white}{black}{\textbf{\textcolor[rgb]{1.00,1.00,1.00}{Example 6.16}}}\hspace*{0.1cm} \texttt{Prove that the following grammar is ambiguous.}\\

\hspace*{4cm} $S \rightarrow 0Y/01$ \\
\hspace*{4cm} $X \rightarrow 0XY/0$ \\
\hspace*{4cm} $Y \rightarrow XY1/1$ \\

\vspace*{0.5cm}
\textbf{Solution: }Take a string 000111. The string can be derived in the following ways.\\

\vspace*{0.3cm}
\begin{flushleft}
  \section{picture}
\includegraphics[width=8cm,height=0.8cm]{307-1.png}
\end{flushleft}
\end{frame}

\begin{frame}
The parse trees for (i) and (ii) are shown is Fig. 6.10.\\

\begin{center}
\section{picture}
\includegraphics[width=8cm,height=5cm]{307-2.png}
\end{center}

\hspace*{0.5cm} As we are getting two parse trees for generating a string from the given grammar, the grammar is
ambiguous.\\
\hspace*{0.5cm} In relation to the ambiguity in CFG, there are few more defi nitions. These are related to a CFL.
These are:\\

\vspace*{0.3cm}
\end{frame}

\begin{frame}
\begin{enumerate}
  \item \textbf{Ambiguous CFL:} A CFG G is said to be ambiguous if there exists some w $\in$ L(G) that has at
least two distinct parse trees.\\
  \item \textbf{Inherently Ambiguous CFL:} A CFL L is said to be inherently ambiguous if all its grammars are
ambiguous.\\
  \item \textbf{Unambiguous CFL:} If L is a CFL for which there exists an unambiguous grammar, then L is said
to be unambiguous.\\
\end{enumerate}

\vspace*{0.3cm}
(Even if one grammar for L is unambiguous, then L is an unambiguous language.)\\
Ambiguous grammar creates problem. Let us take an example.\\
For a grammar G, the production rule is\\

\hspace*{4cm} $E \rightarrow E + E/E*E/a.$ \\

From here, we have to construct $a + a*a$.\\
\end{frame}

\begin{frame}
\section*{Transitional Format}
\begin{flushleft}
    \textbf{308}\hspace*{0.1cm} \textbf{$|$} \hspace*{0.1cm} \texttt{Introduction to Automata Theory, Formal Languages and Computation}
  \end{flushleft}

  \vspace*{0.5cm}
The string can be generated in two different ways\\

 i)  $E \rightarrow E + E \rightarrow E + E*E \rightarrow a + E*E \rightarrow a + a*E \rightarrow a + a*a$ \\
 ii) $E \rightarrow E*E \rightarrow E + E*E \rightarrow a + E*E \rightarrow a + a*E \rightarrow a + a*a$ \\
 
 So, for these two cases two parse trees are generated as shown in Fig. 6.11.\\
 
 \begin{center}
\section{picture}
\includegraphics[width=8cm,height=5cm]{308-1.png}
\end{center}
\end{frame}

\begin{frame}

\small{
So, the grammar is ambiguous.\\
In the place of 'a', put '2'. So, the derivations will be\\
i) $E \rightarrow E + E \rightarrow E + E*E \rightarrow 2 + E*E \rightarrow 2 + 2*E \rightarrow 2 + 2*2$ \\
ii) $E \rightarrow E*E \rightarrow E + E*E \rightarrow 2 + E*E \rightarrow 2 + 2*E \rightarrow 2 + 2*2$ \\
Up to this step, both of them seem the same. But the real problem is in the parse tree as shown in Fig. 6.12.}\\

\begin{center}
\section{picture}
\includegraphics[width=8cm,height=5cm]{308-2.png}
\end{center}

\small{
\hspace*{0.5cm} The correct result is $2 + 2*2 = 2 + 4 = 6$ (according to the rules of mathematics $*$ has higher precedence
over $+$).}\\
\end{frame}
\end{document} 